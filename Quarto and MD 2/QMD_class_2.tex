% Options for packages loaded elsewhere
\PassOptionsToPackage{unicode}{hyperref}
\PassOptionsToPackage{hyphens}{url}
\PassOptionsToPackage{dvipsnames,svgnames,x11names}{xcolor}
%
\documentclass[
  letterpaper,
  DIV=11,
  numbers=noendperiod]{scrartcl}

\usepackage{amsmath,amssymb}
\usepackage{lmodern}
\usepackage{iftex}
\ifPDFTeX
  \usepackage[T1]{fontenc}
  \usepackage[utf8]{inputenc}
  \usepackage{textcomp} % provide euro and other symbols
\else % if luatex or xetex
  \usepackage{unicode-math}
  \defaultfontfeatures{Scale=MatchLowercase}
  \defaultfontfeatures[\rmfamily]{Ligatures=TeX,Scale=1}
\fi
% Use upquote if available, for straight quotes in verbatim environments
\IfFileExists{upquote.sty}{\usepackage{upquote}}{}
\IfFileExists{microtype.sty}{% use microtype if available
  \usepackage[]{microtype}
  \UseMicrotypeSet[protrusion]{basicmath} % disable protrusion for tt fonts
}{}
\makeatletter
\@ifundefined{KOMAClassName}{% if non-KOMA class
  \IfFileExists{parskip.sty}{%
    \usepackage{parskip}
  }{% else
    \setlength{\parindent}{0pt}
    \setlength{\parskip}{6pt plus 2pt minus 1pt}}
}{% if KOMA class
  \KOMAoptions{parskip=half}}
\makeatother
\usepackage{xcolor}
\setlength{\emergencystretch}{3em} % prevent overfull lines
\setcounter{secnumdepth}{2}
% Make \paragraph and \subparagraph free-standing
\ifx\paragraph\undefined\else
  \let\oldparagraph\paragraph
  \renewcommand{\paragraph}[1]{\oldparagraph{#1}\mbox{}}
\fi
\ifx\subparagraph\undefined\else
  \let\oldsubparagraph\subparagraph
  \renewcommand{\subparagraph}[1]{\oldsubparagraph{#1}\mbox{}}
\fi


\providecommand{\tightlist}{%
  \setlength{\itemsep}{0pt}\setlength{\parskip}{0pt}}\usepackage{longtable,booktabs,array}
\usepackage{calc} % for calculating minipage widths
% Correct order of tables after \paragraph or \subparagraph
\usepackage{etoolbox}
\makeatletter
\patchcmd\longtable{\par}{\if@noskipsec\mbox{}\fi\par}{}{}
\makeatother
% Allow footnotes in longtable head/foot
\IfFileExists{footnotehyper.sty}{\usepackage{footnotehyper}}{\usepackage{footnote}}
\makesavenoteenv{longtable}
\usepackage{graphicx}
\makeatletter
\def\maxwidth{\ifdim\Gin@nat@width>\linewidth\linewidth\else\Gin@nat@width\fi}
\def\maxheight{\ifdim\Gin@nat@height>\textheight\textheight\else\Gin@nat@height\fi}
\makeatother
% Scale images if necessary, so that they will not overflow the page
% margins by default, and it is still possible to overwrite the defaults
% using explicit options in \includegraphics[width, height, ...]{}
\setkeys{Gin}{width=\maxwidth,height=\maxheight,keepaspectratio}
% Set default figure placement to htbp
\makeatletter
\def\fps@figure{htbp}
\makeatother

\KOMAoption{captions}{tableheading}
\makeatletter
\makeatother
\makeatletter
\makeatother
\makeatletter
\@ifpackageloaded{caption}{}{\usepackage{caption}}
\AtBeginDocument{%
\ifdefined\contentsname
  \renewcommand*\contentsname{Table of contents}
\else
  \newcommand\contentsname{Table of contents}
\fi
\ifdefined\listfigurename
  \renewcommand*\listfigurename{List of Figures}
\else
  \newcommand\listfigurename{List of Figures}
\fi
\ifdefined\listtablename
  \renewcommand*\listtablename{List of Tables}
\else
  \newcommand\listtablename{List of Tables}
\fi
\ifdefined\figurename
  \renewcommand*\figurename{Figure}
\else
  \newcommand\figurename{Figure}
\fi
\ifdefined\tablename
  \renewcommand*\tablename{Table}
\else
  \newcommand\tablename{Table}
\fi
}
\@ifpackageloaded{float}{}{\usepackage{float}}
\floatstyle{ruled}
\@ifundefined{c@chapter}{\newfloat{codelisting}{h}{lop}}{\newfloat{codelisting}{h}{lop}[chapter]}
\floatname{codelisting}{Listing}
\newcommand*\listoflistings{\listof{codelisting}{List of Listings}}
\makeatother
\makeatletter
\@ifpackageloaded{caption}{}{\usepackage{caption}}
\@ifpackageloaded{subcaption}{}{\usepackage{subcaption}}
\makeatother
\makeatletter
\@ifpackageloaded{tcolorbox}{}{\usepackage[many]{tcolorbox}}
\makeatother
\makeatletter
\@ifundefined{shadecolor}{\definecolor{shadecolor}{rgb}{.97, .97, .97}}
\makeatother
\makeatletter
\makeatother
\ifLuaTeX
  \usepackage{selnolig}  % disable illegal ligatures
\fi
\IfFileExists{bookmark.sty}{\usepackage{bookmark}}{\usepackage{hyperref}}
\IfFileExists{xurl.sty}{\usepackage{xurl}}{} % add URL line breaks if available
\urlstyle{same} % disable monospaced font for URLs
\hypersetup{
  pdftitle={Quarto},
  pdfauthor={Wojciech Hardy},
  colorlinks=true,
  linkcolor={blue},
  filecolor={Maroon},
  citecolor={Blue},
  urlcolor={Blue},
  pdfcreator={LaTeX via pandoc}}

\title{Quarto}
\usepackage{etoolbox}
\makeatletter
\providecommand{\subtitle}[1]{% add subtitle to \maketitle
  \apptocmd{\@title}{\par {\large #1 \par}}{}{}
}
\makeatother
\subtitle{YAML, rendering and parameters}
\author{Wojciech Hardy}
\date{4/20/23}

\begin{document}
\maketitle
\ifdefined\Shaded\renewenvironment{Shaded}{\begin{tcolorbox}[borderline west={3pt}{0pt}{shadecolor}, breakable, sharp corners, enhanced, interior hidden, frame hidden, boxrule=0pt]}{\end{tcolorbox}}\fi

\listoffigures
\listoftables
\hypertarget{moving-from-different-formats-to-quarto}{%
\section{Moving from different formats to
Quarto}\label{moving-from-different-formats-to-quarto}}

\hypertarget{migrating-from-rmarkdown}{%
\subsection{Migrating from RMarkdown}\label{migrating-from-rmarkdown}}

Quarto is quite new. It's enough similar to RMarkdown to make most of
the uses convertible. How?

From a regular R script:

Step 1) Run
\texttt{knitr::convert\_chunk\_header(input\ =\ "RMD\_class\_1\_cut.Rmd",\ output\ =\ "QMD\_class\_1\_cut.qmd")}

Step 2) Edit the YAML options (in notepad, through R, etc.)

\begin{longtable}[]{@{}ll@{}}
\toprule()
RMD & Quarto \\
\midrule()
\endhead
output: & format: \\
output: html\_document & format: html \\
output: pdf\_document & format: pdf \\
output: word\_document & format: docx \\
output: xaringan & format: revealjs \\
output: ioslides & format: ioslides \\
: \href{https://quarto.org/docs/faq/rmarkdown.html}{Other} : & \\
\bottomrule()
\end{longtable}

Done!

\hypertarget{turning-your-notebook-into-a-quarto-document}{%
\subsection{Turning your Notebook into a Quarto
document}\label{turning-your-notebook-into-a-quarto-document}}

Step 1) insert a YAML header in a \emph{raw} block

For example:

\texttt{-\/-\/-}

\texttt{-\/-\/-}

Done! You don't even need to switch to *.qmd!

But if you want embedded code chunks, you should switch to .qmd.

You can convert a *.ipynb file to a *.qmd file and \emph{vice versa}
with: \texttt{quarto\ convert\ notebook.ipynb} (to qmd)
\texttt{quarto\ convert\ notebook.qmd} (to ipynb)

Or \href{https://jupytext.readthedocs.io/en/latest/}{use
\texttt{Jupytext} to have both (synced)}.

\begin{center}\rule{0.5\linewidth}{0.5pt}\end{center}

\hypertarget{to-add-a-refreshable-preview}{%
\section{To add a (refreshable)
preview}\label{to-add-a-refreshable-preview}}

From terminal run:

\texttt{quarto\ preview\ file.qmd}

or

\texttt{quarto\ preview\ notebook.ipynb}

(You can preview in pdf by adding \texttt{-\/-to\ pdf})

\begin{center}\rule{0.5\linewidth}{0.5pt}\end{center}

\hypertarget{need-a-recap}{%
\section{Need a recap?}\label{need-a-recap}}

Remember you can
\href{https://quarto.org/docs/authoring/markdown-basics.html}{check the
documentation} - there's many more options than we covered.

For example, let's check here for some more info on
\href{https://quarto.org/docs/authoring/figures.html}{figures} or
\href{https://quarto.org/docs/authoring/diagrams.html}{diagrams} or
other \href{https://quarto.org/docs/authoring/notebook-embed.html}{neat
tricks}.

\begin{center}\rule{0.5\linewidth}{0.5pt}\end{center}

\hypertarget{so-what-is-yaml}{%
\section{So what is YAML?}\label{so-what-is-yaml}}

\textbf{\href{https://yaml.org/}{YAML}} = \textbf{Y}AML \textbf{A}in't
\textbf{M}arkup \textbf{L}anguage

As Wikipedia puts it: ``a human-readable data-serialization language''.

If that's not clear, ChatGPT makes it clearer:

\begin{enumerate}
\def\labelenumi{\arabic{enumi})}
\item
  a way to represent computer data in a format that is easily understood
  by humans, allowing for easier interpretation and modification of data
  without needing specialized software or tools
\item
  these languages are commonly used in web development, data exchange
  between different systems, and configuration files for applications.
\end{enumerate}

\begin{center}\rule{0.5\linewidth}{0.5pt}\end{center}

\hypertarget{yaml-in-quarto---options}{%
\section{YAML in Quarto - options}\label{yaml-in-quarto---options}}

\hypertarget{basics}{%
\subsection{Basics}\label{basics}}

\hypertarget{the-syntax}{%
\subsubsection{The syntax}\label{the-syntax}}

YAML requires indentation for options and suboptions (accordingly with
the hierarchy). Examples from leaset complex to most complex:

\texttt{-\/-\/-}

\texttt{-\/-\/-}

Now with html-specific options: \texttt{-\/-\/-}

\texttt{-\/-\/-}

\hypertarget{banners}{%
\subsubsection{Banners}\label{banners}}

You can highlight the title with a banner:
\texttt{title-block-banner:\ true}

or include an image for a background:
\texttt{title-block-banner:\ images/banner.jpeg}

\hypertarget{dates}{%
\subsubsection{Dates}\label{dates}}

Change the date format and make it dynamic:

\texttt{date:\ today} \texttt{date-format:\ short}

\hypertarget{setting-default-chunk-options}{%
\subsubsection{Setting default chunk
options}\label{setting-default-chunk-options}}

Within the \texttt{execute} key you can set default options for code
blocks. They can be overridden within the specific code chunks.

Example:

\texttt{-\/-\/-} execute: echo: fenced warning: false cache: true
freeze: true \texttt{-\/-\/-}

\texttt{freeze} specifies a more general approach for caching when
rendering a bunch of files or a whole project. \texttt{true} to not
recompute generated computational output during a whole project render.
\texttt{false} to always recompute (default). \texttt{auto} to recompute
when a source file changed.

\hypertarget{switching-formats}{%
\subsubsection{Switching formats}\label{switching-formats}}

Let's try \texttt{pdf} and \texttt{docx}.

\href{https://quarto.org/docs/output-formats/all-formats.html}{For all
available formats check here.}

\hypertarget{html-options}{%
\subsection{HTML options}\label{html-options}}

The html format is the most flexible as it enables dynamic and
interactive elements.

To get an html final output, code in .qmd gets evaluated and the output
is saved to .md (regular markdown). This is then translated using pandoc
to html.

\hypertarget{table-of-contents}{%
\subsubsection{Table of Contents}\label{table-of-contents}}

\texttt{-\/-\/-}

\texttt{-\/-\/-}

\texttt{toc-depth} describes the levels included (default 3).

\texttt{toc-expand} how many should be expanded from the start
(\texttt{true} if all, default \texttt{false} for none).

\texttt{toc-title} to add a caption.

\texttt{toc-location} to move the ToC (\texttt{left}, \texttt{right} or
\texttt{body})

\texttt{smooth-scroll:\ true} to enable smooth scrolling.

\hypertarget{section-numbering}{%
\subsubsection{Section numbering}\label{section-numbering}}

\texttt{number-sections:\ true}

\texttt{number-depth:\ 3} Number of levels for numbering.

\hypertarget{global-code-chunk-options}{%
\subsubsection{Global code-chunk
options}\label{global-code-chunk-options}}

\texttt{code-fold} to make it \textbf{optional} whether to display code.
\texttt{true} to start with collapsed, \texttt{false} to not include,
\texttt{show} to start with shown but collapsable.

\texttt{code-summary} specify text for collapsed code.

\hypertarget{including-html-elements}{%
\subsubsection{Including html elements}\label{including-html-elements}}

E.g.:

\hypertarget{styling}{%
\subsubsection{Styling}\label{styling}}

You can use css for styling:

\hypertarget{themes}{%
\subsubsection{Themes}\label{themes}}

You can do a quick swap of the whole theme by using one of 25 themes
from \href{https://bootswatch.com/}{Bootswatch}. Try:

\texttt{theme:\ lumen}

\texttt{theme:\ solar}

\texttt{theme:\ superhero}

You can also provide a dark and light option:

\hypertarget{some-others-not-all}{%
\subsubsection{Some others (not all)}\label{some-others-not-all}}

Disable the default special features with:

\texttt{minimal:\ true}

\hypertarget{divs-and-spans}{%
\subsubsection{Divs and Spans}\label{divs-and-spans}}

You can use classes and define `div' regions (as in html nodes), by
indicating the start and end of a region with \texttt{:::}.

For example:

\hypertarget{putting-your-stuff-in-tabs}{%
\subsubsection{Putting your stuff in
tabs}\label{putting-your-stuff-in-tabs}}

For example:

\hypertarget{lists}{%
\section{Lists}\label{lists}}

\hypertarget{ordered}{%
\paragraph{Ordered}\label{ordered}}

\texttt{1.\ Item\ 1}

\texttt{2.\ Item\ 2}

\texttt{2.\ Item\ 3\ \#\ Note\ the\ error\ in\ numbering}

\begin{enumerate}
\def\labelenumi{\arabic{enumi}.}
\item
  Item 1
\item
  Item 2
\item
  Item 3 \texttt{\#\ It\textquotesingle{}s\ fine\ here\ though}
\end{enumerate}

\hypertarget{unordered}{%
\paragraph{Unordered}\label{unordered}}

\texttt{*\ Item}

\texttt{*\ Another\ item}

\begin{itemize}
\item
  Item
\item
  Another item
\end{itemize}

\hypertarget{subitems}{%
\paragraph{Subitems}\label{subitems}}

\texttt{1.\ \ Item\ 1}

\texttt{-\ \ \ Item\ 2}

\texttt{-\ \ \ Item\ 3}

\begin{enumerate}
\def\labelenumi{\arabic{enumi}.}
\tightlist
\item
  Item 1

  \begin{itemize}
  \tightlist
  \item
    Item 2
  \item
    Item 3
  \end{itemize}
\end{enumerate}

\hypertarget{creating-a-self-contained-html}{%
\subsubsection{Creating a self-contained
Html}\label{creating-a-self-contained-html}}

\texttt{-\/-\/-}

\texttt{-\/-\/-}

\hypertarget{pdf-options}{%
\subsection{PDF options}\label{pdf-options}}

Many of the options for the pdf and docx formats look the same. Main
differences:

\begin{enumerate}
\def\labelenumi{\arabic{enumi})}
\item
  no dynamic elements (e.g.~no floating table of content)
\item
  no direct html support
\item
  TeX support (e.g.~you can use document classes, etc.)
\end{enumerate}

\hypertarget{troubleshooting}{%
\subsubsection{Troubleshooting}\label{troubleshooting}}

Getting from markdown to pdf requires a TeX installation. Quarto will
propose installing \texttt{tinytex} via its tools. You can try that.

If it's not working, troubleshooting usually goes through some
reinstallation.

You can also try adding:

\texttt{latex-tinytex:\ false}

this will force Quarto to use what you have installed in the system
instead of the Quarto installation.

\hypertarget{including-elements}{%
\subsubsection{Including elements}\label{including-elements}}

Instead of html files and code, you may attach .tex files and TeX
writing:

\texttt{-\/-\/-}

\texttt{-\/-\/-}

\hypertarget{lists-of-tables-and-figures}{%
\subsubsection{Lists of tables and
figures}\label{lists-of-tables-and-figures}}

\texttt{lof:\ true}

\texttt{lot:\ true}

\hypertarget{combining-formats}{%
\subsection{Combining formats}\label{combining-formats}}

You can also render several formats and link them together.

For example:

\texttt{-\/-\/-}

\texttt{-\/-\/-}

Or if the file extensions overlap:

\texttt{-\/-\/-}

\texttt{-\/-\/-}

\hypertarget{getting-an-.md-file}{%
\subsection{Getting an .md file}\label{getting-an-.md-file}}

We might be interested in getting a pure .md file (for example for a
readme.md file for GitHub).

.md is a byproduct of the rendering process, usually deleted by the end
of the process.

To keep it:

\texttt{keep-md:\ true}

Try it for the \texttt{QMD\_class\_1\_cut.qmd} file and compare QMD with
MD.

\hypertarget{exporting-yaml}{%
\section{Exporting YAML}\label{exporting-yaml}}

We've been editing YAML before. We can actually put the info from YAML
in a separate file that will be automatically considered for all qmd
documents.

It's helpful when:

\begin{itemize}
\tightlist
\item
  there's several documents that we want to have similar settings.
\item
  we work with a team and want everyone to use the same settings.
\end{itemize}

We have several options here:

\texttt{\_quarto.yml} file keeps information at a project level

\texttt{dir/\_metadata.yml} file keeps information at a directory level
(e.g.~within a project)

\texttt{dir/document.qmd} is just what we put in the document header.

Document overrides \texttt{\_metadata.yml} which overrides
\texttt{\_quarto.yml}.

You can create the .yml files in notepad.

Mostly the options are merged rather than overwritten (so bundled
together).

\hypertarget{rendering}{%
\section{Rendering}\label{rendering}}

You can render Quarto MD files from your scripts or terminal. To do so,
use:

\hypertarget{terminal}{%
\subsection{Terminal}\label{terminal}}

For a whole project:

\texttt{quarto\ render} in current folder or
\texttt{quarto\ render\ myproject} for the \texttt{myproject} project.

For a file:

\texttt{quarto\ render\ document.qmd}

For a folder:

\texttt{quarto\ render\ directory/}

\hypertarget{from-the-level-of-r}{%
\subsection{From the level of R}\label{from-the-level-of-r}}

\texttt{library(quarto)} \texttt{quarto\_render()}

You can specify options for the rendering, including:

\texttt{output\_format}, \texttt{output\_file}, \texttt{use\_freezer},
\texttt{execute\_params}, etc.

Parameters will let you specify data inputs at the document's
conception.

\hypertarget{assignment}{%
\section{Assignment}\label{assignment}}

Get your TV show description file from the previous week. Modify it
using what we've learned today:

\begin{enumerate}
\def\labelenumi{\arabic{enumi})}
\item
  Render it into at least two different formats in an automated way
  (e.g.~using YAML options or a script).
\item
  Add some features for the .html output, e.g.: add a table of contents
  and change the theme.
\item
  Put the YAML options in a separate file.
\item
  Include tabs.
\item
  Browse through the documentation and pick additional configuration
  options on your own.
\end{enumerate}

Note: if you didn't finish the assignment last week, you can work with
the Assignment.Rmd file. You'll have to convert it to .qmd first though
(use what we've learned today for that!)



\end{document}
