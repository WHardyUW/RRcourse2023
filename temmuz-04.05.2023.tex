% Options for packages loaded elsewhere
\PassOptionsToPackage{unicode}{hyperref}
\PassOptionsToPackage{hyphens}{url}
\PassOptionsToPackage{dvipsnames,svgnames,x11names}{xcolor}
%
\documentclass[
  letterpaper,
  DIV=11,
  numbers=noendperiod]{scrartcl}

\usepackage{amsmath,amssymb}
\usepackage{lmodern}
\usepackage{iftex}
\ifPDFTeX
  \usepackage[T1]{fontenc}
  \usepackage[utf8]{inputenc}
  \usepackage{textcomp} % provide euro and other symbols
\else % if luatex or xetex
  \usepackage{unicode-math}
  \defaultfontfeatures{Scale=MatchLowercase}
  \defaultfontfeatures[\rmfamily]{Ligatures=TeX,Scale=1}
\fi
% Use upquote if available, for straight quotes in verbatim environments
\IfFileExists{upquote.sty}{\usepackage{upquote}}{}
\IfFileExists{microtype.sty}{% use microtype if available
  \usepackage[]{microtype}
  \UseMicrotypeSet[protrusion]{basicmath} % disable protrusion for tt fonts
}{}
\makeatletter
\@ifundefined{KOMAClassName}{% if non-KOMA class
  \IfFileExists{parskip.sty}{%
    \usepackage{parskip}
  }{% else
    \setlength{\parindent}{0pt}
    \setlength{\parskip}{6pt plus 2pt minus 1pt}}
}{% if KOMA class
  \KOMAoptions{parskip=half}}
\makeatother
\usepackage{xcolor}
\setlength{\emergencystretch}{3em} % prevent overfull lines
\setcounter{secnumdepth}{-\maxdimen} % remove section numbering
% Make \paragraph and \subparagraph free-standing
\ifx\paragraph\undefined\else
  \let\oldparagraph\paragraph
  \renewcommand{\paragraph}[1]{\oldparagraph{#1}\mbox{}}
\fi
\ifx\subparagraph\undefined\else
  \let\oldsubparagraph\subparagraph
  \renewcommand{\subparagraph}[1]{\oldsubparagraph{#1}\mbox{}}
\fi


\providecommand{\tightlist}{%
  \setlength{\itemsep}{0pt}\setlength{\parskip}{0pt}}\usepackage{longtable,booktabs,array}
\usepackage{calc} % for calculating minipage widths
% Correct order of tables after \paragraph or \subparagraph
\usepackage{etoolbox}
\makeatletter
\patchcmd\longtable{\par}{\if@noskipsec\mbox{}\fi\par}{}{}
\makeatother
% Allow footnotes in longtable head/foot
\IfFileExists{footnotehyper.sty}{\usepackage{footnotehyper}}{\usepackage{footnote}}
\makesavenoteenv{longtable}
\usepackage{graphicx}
\makeatletter
\def\maxwidth{\ifdim\Gin@nat@width>\linewidth\linewidth\else\Gin@nat@width\fi}
\def\maxheight{\ifdim\Gin@nat@height>\textheight\textheight\else\Gin@nat@height\fi}
\makeatother
% Scale images if necessary, so that they will not overflow the page
% margins by default, and it is still possible to overwrite the defaults
% using explicit options in \includegraphics[width, height, ...]{}
\setkeys{Gin}{width=\maxwidth,height=\maxheight,keepaspectratio}
% Set default figure placement to htbp
\makeatletter
\def\fps@figure{htbp}
\makeatother

\KOMAoption{captions}{tableheading}
\makeatletter
\makeatother
\makeatletter
\makeatother
\makeatletter
\@ifpackageloaded{caption}{}{\usepackage{caption}}
\AtBeginDocument{%
\ifdefined\contentsname
  \renewcommand*\contentsname{Table of contents}
\else
  \newcommand\contentsname{Table of contents}
\fi
\ifdefined\listfigurename
  \renewcommand*\listfigurename{List of Figures}
\else
  \newcommand\listfigurename{List of Figures}
\fi
\ifdefined\listtablename
  \renewcommand*\listtablename{List of Tables}
\else
  \newcommand\listtablename{List of Tables}
\fi
\ifdefined\figurename
  \renewcommand*\figurename{Figure}
\else
  \newcommand\figurename{Figure}
\fi
\ifdefined\tablename
  \renewcommand*\tablename{Table}
\else
  \newcommand\tablename{Table}
\fi
}
\@ifpackageloaded{float}{}{\usepackage{float}}
\floatstyle{ruled}
\@ifundefined{c@chapter}{\newfloat{codelisting}{h}{lop}}{\newfloat{codelisting}{h}{lop}[chapter]}
\floatname{codelisting}{Listing}
\newcommand*\listoflistings{\listof{codelisting}{List of Listings}}
\makeatother
\makeatletter
\@ifpackageloaded{caption}{}{\usepackage{caption}}
\@ifpackageloaded{subcaption}{}{\usepackage{subcaption}}
\makeatother
\makeatletter
\@ifpackageloaded{tcolorbox}{}{\usepackage[many]{tcolorbox}}
\makeatother
\makeatletter
\@ifundefined{shadecolor}{\definecolor{shadecolor}{rgb}{.97, .97, .97}}
\makeatother
\makeatletter
\makeatother
\ifLuaTeX
  \usepackage{selnolig}  % disable illegal ligatures
\fi
\IfFileExists{bookmark.sty}{\usepackage{bookmark}}{\usepackage{hyperref}}
\IfFileExists{xurl.sty}{\usepackage{xurl}}{} % add URL line breaks if available
\urlstyle{same} % disable monospaced font for URLs
\hypersetup{
  pdftitle={Quarto},
  pdfauthor={Temmuz Yavuzer},
  colorlinks=true,
  linkcolor={blue},
  filecolor={Maroon},
  citecolor={Blue},
  urlcolor={Blue},
  pdfcreator={LaTeX via pandoc}}

\title{Quarto}
\usepackage{etoolbox}
\makeatletter
\providecommand{\subtitle}[1]{% add subtitle to \maketitle
  \apptocmd{\@title}{\par {\large #1 \par}}{}{}
}
\makeatother
\subtitle{YAML, rendering and parameters}
\author{Temmuz Yavuzer}
\date{5/18/23}

\begin{document}
\maketitle
\ifdefined\Shaded\renewenvironment{Shaded}{\begin{tcolorbox}[breakable, borderline west={3pt}{0pt}{shadecolor}, frame hidden, boxrule=0pt, interior hidden, enhanced, sharp corners]}{\end{tcolorbox}}\fi

\hypertarget{game-of-thrones}{%
\section{Game of Thrones}\label{game-of-thrones}}

\textbf{\emph{Game of Thrones}} is an American
\href{https://en.wikipedia.org/wiki/Fantasy_television}{fantasy}
\href{https://en.wikipedia.org/wiki/Drama_(film_and_television)}{drama}
television series created by
\href{https://en.wikipedia.org/wiki/David_Benioff}{David Benioff} and
\href{https://en.wikipedia.org/wiki/D._B._Weiss}{D. B. Weiss} for
\href{https://en.wikipedia.org/wiki/HBO}{HBO}. It is an adaptation of
\href{https://en.wikipedia.org/wiki/A_Song_of_Ice_and_Fire}{\emph{A Song
of Ice and Fire}}, a series of fantasy novels by
\href{https://en.wikipedia.org/wiki/George_R._R._Martin}{George R. R.
Martin}, the first of which is
\href{https://en.wikipedia.org/wiki/A_Game_of_Thrones}{\emph{A Game of
Thrones}}. The show was shot in the United Kingdom, Canada, Croatia,
Iceland, Malta, Morocco, and Spain. It premiered on HBO in the United
States on April 17, 2011, and concluded on May 19, 2019, with 73
episodes broadcast over eight seasons.

\includegraphics{https://upload.wikimedia.org/wikipedia/en/thumb/d/d8/Game_of_Thrones_title_card.jpg/250px-Game_of_Thrones_title_card.jpg}

\hypertarget{ratings}{%
\subsection{Ratings}\label{ratings}}

Season \textbar{} Episode\_number \textbar{} Raiting \textbar{}

\textbar:-------\textbar:--------------:\textbar--------:\textbar{}

1 \textbar{} 1 \textbar{} 9 \textbar{}

3 \textbar{} 1 \textbar{} 9.30 \textbar{}

4 \textbar{} 1 \textbar{} 9.52 \textbar{}

\hypertarget{header-1}{%
\paragraph{Header 1}\label{header-1}}

Text with headers

\hypertarget{header-2}{%
\paragraph{Header 2}\label{header-2}}

Some more text

\hypertarget{graphs}{%
\subsection{Graphs}\label{graphs}}

\hypertarget{quarto}{%
\subsection{Quarto}\label{quarto}}

Quarto enables you to weave together content and executable code into a
finished document. To learn more about Quarto see
\url{https://quarto.org}.

\hypertarget{running-code}{%
\subsection{Running Code}\label{running-code}}

When you click the Render button a document will be generated that
includes both content and the output of embedded code. You can embed
code like this:

\{r\}

1 + 1

You can add options to executable code like this

\{r\}

\#\textbar{} echo: false

2 * 2



\end{document}
